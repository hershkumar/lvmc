% arara: pdflatex: { shell : yes }
% % arara: bibtex
% arara: pdflatex if found('log', 'undefined references')


\documentclass{notes}

\usepackage{amsmath}
\usepackage{mathtools}
\usepackage{braket}
\usepackage{tcolorbox}
\usepackage{amssymb}
\usepackage{amsthm}
\usepackage{listings}
\usepackage{xcolor}
\usepackage{bm}
\usepackage{esint}
\usepackage{gensymb}
\usepackage{mathrsfs}

\usepackage{slashed}
\usepackage{simpler-wick}
\usepackage{tikz}
\usepackage{tikz-feynman}
\usepackage{tikz-feynhand}

\definecolor{codegreen}{rgb}{0,0.6,0}
\definecolor{codegray}{rgb}{0.5,0.5,0.5}
\definecolor{codepurple}{rgb}{0.58,0,0.82}
\definecolor{backcolour}{rgb}{0.95,0.95,0.92}

\lstdefinestyle{mystyle}{
    backgroundcolor=\color{backcolour},   
    commentstyle=\color{codegreen},
    keywordstyle=\color{magenta},
    numberstyle=\tiny\color{codegray},
    stringstyle=\color{codepurple},
    basicstyle=\ttfamily\footnotesize,
    breakatwhitespace=false,         
    breaklines=true,                 
    captionpos=b,                    
    keepspaces=true,                 
    numbers=left,                    
    numbersep=5pt,                  
    showspaces=false,                
    showstringspaces=false,
    showtabs=false,                  
    tabsize=2
}

\lstset{style=mystyle}


\newtheorem{postulate}{Postulate}
\newtheorem{theorem}{Theorem}[section]
\newtheorem{corollary}{corollary}[section]
\newtheorem{lemma}[theorem]{Lemma}

\title{Sigma Model Lattice VMC}
\date{}
\author{}

\let\ve\varepsilon
\let\vec\bm
\begin{document}
\maketitle

The 1D $\sigma$ model Hamiltonian is given by:
 \begin{align*}
	 \hat{H} &=  \eta g^2 \sum_{i=0}^N  \vec{L}^2_i + \frac{\eta}{g^2}\sum_{i=0}^N \hat{n}_i \cdot \hat{n}_{i+1}
\end{align*}
Where $\hat n_i$ is parameterized by two angles,  $\rho_i$ and  $\phi_i$.
We define our ansatz as:
\begin{align*}
	\psi\left(\{\rho_i, \phi_i\}\right) &=  e^{-\text{NN}\left(\{\rho_i, \phi_i\}\right)}
\end{align*}

We define the energy in the usual VMC way:
\begin{align*}
	\mathcal{E} &=  \left\langle \frac{H\psi}{\psi}\right\rangle_{\psi^2}
\end{align*}
And the gradient in the usual way:
\begin{align*}
\frac{\partial \mathcal{E}}{\partial \theta} &=  2 \left\langle \frac{\partial \ln\psi}{\partial \theta} \frac{H\psi}{\psi}\right\rangle - \left\langle \frac{\partial
\ln\psi}{\partial \theta} \right\rangle \left\langle \frac{H\psi}{\psi} \right\rangle
\end{align*}




\end{document}
